\documentclass{article}
\usepackage[utf8]{inputenc}

\usepackage{amsmath}
\usepackage{amssymb}
\usepackage{amsthm}

\title{}
\author{Adam Fiedler}

\begin{document}

\maketitle

Let $f, g : \mathbb{N} \rightarrow \mathbb{N}$ and $f \in \mathcal{O}(g)$. Then

\[ \mathcal{O}(f) \subseteq \mathcal{O}(g). \]

\begin{proof}

Suppose that $f \in \mathcal{O}(g)$, therefore by definition exists $n_0, c_0 \in \mathbb{N}$ such that for all $n \geq n_0 \in \mathbb{N}$ holds 

\[f(n) \leq c_0 * g(n).\]

From this assumption follows that for some $c_1, n_1 \in \mathbb{N}$ and all $n \geq n_0 + n_1$ must hold

\[ \mathcal{O}(f) = \{ h \mid h(n) \leq c_1 * f(n) \leq c_1 * (c_0 * g(n)) \}. \]

Now let us consider any $h \in \mathcal{O}(f)$, then we know from the above proposition about $\mathcal{O}(f)$ that there exists $c', n' \in \mathbb{N}$ such that for all $n \geq n'$ holds

\[ h(n) \leq c' * g(n), \]

\noindent namely $c' = c_1 * c_0$ and $n' = n_0 + n_1$. But this means by definition that $h \in \mathcal{O}(g)$, therefore $\mathcal{O}(f) \subseteq \mathcal{O}(g)$.

\end{proof}

\end{document}
