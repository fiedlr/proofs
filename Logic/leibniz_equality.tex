\documentclass{article}
\usepackage[utf8]{inputenc}

\usepackage{amsmath}
\usepackage{amssymb}
\usepackage{amsthm}

\title{}
\author{Adam Fiedler}

\begin{document}

\maketitle

Let us look at a nice proposition called \textit{Leibniz equality}:

\[ x = y \iff \forall P . (P(x) \iff P(y)), \]

\noindent where $P$ is a \emph{predicate}. If we do not take it as the definition of equality itself, it can be proven using the usual laws of logic and a little trick.

\begin{proof}
We shall prove it by proving both implications.
\begin{enumerate}
  \item ($\Rightarrow$) Trivial.
  \item ($\Leftarrow$) If every predicate holds for $x$ if and only if it holds for $y$, then specifically this must apply also if we set $P$ to the \emph{equality test with x} ($P := (= x)$). Trivially $x = x$, using the assumption therefore $y = x$. Equality being symmetric, from this follows $x = y$.
\end{enumerate}

\end{proof}

\end{document}
