\documentclass{article}
\usepackage[utf8]{inputenc}

\usepackage{amsmath}
\usepackage{amssymb}
\usepackage{amsthm}

\title{}
\author{Adam Fiedler}

\begin{document}

\maketitle

We shall want to prove using $\sim_L$ that given any two languages $L, K$ over any finite alphabet, while supposing $K$ is finite (i.e. $K$ contains a finite number of words), the following holds:

\[L \cup K \text{ is regular} \implies L \text{ is regular}\]

\begin{proof}

Let's prove the contrapositive of the statement.

Suppose that $L$ is not regular, from \emph{Myhill-Nerod theorem} then follows that $\sim_L$ has an infinite index. In other words, there is an infinite number of different words over the given alphabet such that they are not equivalent in terms of $\sim_L$. Let's call this set of words $M$.\\

We know that $K$ is finite and without loss of generality, let's suppose it is nonempty (otherwise the theorem would be trivial to prove). From this follows there must exist a word $w \in K$ such that $(\forall k \in K)(|k| \leq |w|)$.\\ 

Consider the nonempty subset of $M \supseteq N = \{\forall m \in M: |m| > |w|\}$. All words in $N$ clearly are not member of $K$ whether we extend them with any suffix or not (the suffix used will be $\epsilon$), hence from the definition of $\sim_{L \cup K}$ for all $u, v \in N$ holds that $u \sim_{L \cup K} v \iff u \sim_{L} v$.\newline.

However, we've supposed that $u \not\sim_L v$ for $u \neq v$ because $L$ is not regular. This means we've found a countably infinite set of words $N$, which are not equivalent in terms of $\sim_{L \cup K}$, therefore $L \cup K$ is not regular.

\end{proof}

\end{document}
