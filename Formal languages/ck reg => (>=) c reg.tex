\documentclass{article}
\usepackage[utf8]{inputenc}

\usepackage{amsmath}
\usepackage{amssymb}
\usepackage{amsthm}

\title{}
\author{Adam Fiedler}

\begin{document}

\maketitle

Let $c \in \mathbb{N}$, $\Sigma$ is any finite alphabet, $L = \{w \in \Sigma^*: |w| = ck \mid k \in \mathbb{N}\}, K = \{w \in \Sigma^*: |w| \geq c\}$. Then the following holds:

\[L \text{ is regular} \implies K \text{ is regular}\]

\begin{proof}
For $c = 1$ the proof is trivial ($L = K$). For $c > 1$ let us call the languages of remainder classes modulo $c$ as $L_{0}$, $L_{1}$, ..., $L_{c-1}$. We shall thus consider a system of languages $L_i = \{w \in \Sigma^*: |w| = ck + i \mid k \in \mathbb{N}\} \subset K$, where $i \in I = \{0, 1, ..., c-1\} \subset \mathbb{N}_0$. $K$ can then be written as:

\[ K = \bigcup\limits_{i \in I} L_{i} \]

\noindent Moreover the following lemma holds for all $m, n \in I, m \leq n$:

\[ L_m\text{ is regular} \implies L_n\text{ is regular} \]

\noindent This can be easily shown because we know that the class of regular languages is closed under concatenation and clearly $L_m \cdot \Sigma^{n - m} = L_n$.\newline

Finally, based on the premise about the regularity of $L = L_0$, from the aforementioned lemma follows that languages $L_i$ are regular. And because we know that regular languages are closed under union, hence also the big union $K$ is regular.

\end{proof}

\end{document}
