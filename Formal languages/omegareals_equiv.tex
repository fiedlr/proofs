\documentclass{article}
\usepackage[utf8]{inputenc}
\usepackage[shortlabels]{enumitem}

\usepackage{amsmath}
\usepackage{amssymb}
\usepackage{amsthm}

\title{}
\author{Adam Fiedler}

\begin{document}

\maketitle

Let $\Sigma = \{\mathbf{0}, \mathbf{1}, \mathbf{2}, \mathbf{3}, \mathbf{4}, \ldots, \mathbf{9}\}$ and $\mathbf{R} = \Sigma^\omega$, then:

\[ | \mathbf{R} | = | \mathbb{R} |. \]

\begin{proof}
We know that $|\mathbb{R}| = |[0, 1)|$, therefore it is sufficient to prove $| [0, 1) | = | \mathbf{R} |$.
Let us consider the function $f : [0, 1) \to \mathbf{R}$ defined as follows

\[f(0.x_0x_1x_2x_3\ldots) = t(x_0)t(x_1)t(x_2)t(x_3)\ldots\]

\noindent where $t(0) = \mathbf{0}, t(1) = \mathbf{1}, \ldots, t(9) = \mathbf{9}$. Then $f$ is a \emph{bijection} because 

\begin{enumerate}[a)]
\item $f$ is injective. This is obvious.
\item $f$ is surjective. 
We can prove by contradiction. 
Suppose that there is such $y = a_0a_1a_2\ldots \in \mathbf{R}$ that there is no $x \in [0, 1)$ satisfying $f(x) = y$. 
By definition of $f$, this can only be for some $a_i \in \Sigma$ such that for no $z$ holds $t(z) = a_i$. 
However, this is in contradiction with $range(t) = \Sigma$.
\end{enumerate}

\end{proof}
\end{document}
