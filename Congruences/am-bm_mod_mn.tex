\documentclass{article}
\usepackage[utf8]{inputenc}

\usepackage{amsmath}
\usepackage{amssymb}
\usepackage{amsthm}

\title{}
\author{Adam Fiedler}

\begin{document}

\maketitle

Let $m, n \in \mathbb{N}$ and $a, b \in \mathbb{Z}$. Then

\[ a \equiv b \text{ (mod $m^n$)} \implies a^m \equiv b^m \text{ (mod $m^{n+1}$)}  \]

\begin{proof}

What we want actually to prove is just this. Supposing $m^n \mid a - b$, then also $m^{n+1} \mid a^m - b^m$. The reader can verify for herself that $a - b$ can be factored out of $a^m - b^m$ as follows:

\[ a^m - b^m = (a - b) \cdot P(a, b) \]

where $P(a, b)$ is a polynomial expression containing $a, b$:

\[ P(a, b) = a^{m-1} + a^{m-2}b + ... + ab^{m-2} + b^{m-1}  \]

From our premise we know that $m^n \mid a - b$, therefore we only have to show that $m \mid P(a, b)$. This is because it would mean that $a^m - b^m$ is a product of multiples of $m^n$ and $m$, in other words, dividible by $m^{n+1}$.

Consider that the supposed congruence by definition also tells us $a, b$ gives the same remainder after being divided by $m$. Hence $a = km^n + r$ and $b = lm^n + r$ for some $k, l \in \mathbb{Z}$ and $r \in \mathbb{N}$. 

Thus we can factor out $m$ out of every member of $P$ besides $r^{m-1}$, which is obviously present in $P$ exactly $m$ times. Therefore for some $l, k \in \mathbb{Z}$ holds:

\[ P(a, b) = m \cdot l + m \cdot r^{m-1} = m \cdot k \]

\end{proof}

\end{document}
